%%%%%%%%%%%%%%%%%%%%%%%%%%%%%%%%%%%%%%%%%%%%%%%%%%%%%%%%%%%%%%%%%%%%%
% LaTeX Template: Project Titlepage Modified (v 0.1) by rcx
%
% Original Source: http://www.howtotex.com
% Date: February 2014
% 
% This is a title page template which be used for articles & reports.
% 
% This is the modified version of the original Latex template from
% aforementioned website.
% 
%%%%%%%%%%%%%%%%%%%%%%%%%%%%%%%%%%%%%%%%%%%%%%%%%%%%%%%%%%%%%%%%%%%%%%

\documentclass[12pt]{report}
\usepackage[a4paper]{geometry}
\usepackage[utf8]{inputenc}
\usepackage[myheadings]{fullpage}
\usepackage{fancyhdr}
\usepackage{lastpage}
\usepackage{graphicx, wrapfig, subcaption, setspace, booktabs}
\usepackage[T1]{fontenc}
\usepackage[font=small, labelfont=bf]{caption}
\usepackage{fourier}
\usepackage[protrusion=true, expansion=true]{microtype}
\usepackage[english]{babel}
\usepackage{sectsty}
\usepackage{url, lipsum}
\usepackage{listings}
\usepackage{hyperref}

\usepackage{color}

\definecolor{codegreen}{rgb}{0,0.6,0}
\definecolor{codegray}{rgb}{0.5,0.5,0.5}
\definecolor{codepurple}{rgb}{0.58,0,0.82}
\definecolor{backcolour}{rgb}{0.95,0.95,0.92}

\lstdefinestyle{codigo}{
    backgroundcolor=\color{backcolour},   
    commentstyle=\color{codegreen},
    keywordstyle=\color{magenta},
    numberstyle=\tiny\color{codegray},
    stringstyle=\color{codepurple},
    basicstyle=\footnotesize,
    breakatwhitespace=false,         
    breaklines=true,                 
    captionpos=b,                    
    keepspaces=true,                 
    numbers=left,                    
    numbersep=3pt,                  
    showspaces=false,                
    showstringspaces=false,
    showtabs=false,                  
    tabsize=2
}

\lstset{style=codigo}

\newcommand{\HRule}[1]{\rule{\linewidth}{#1}}
\onehalfspacing
\setcounter{tocdepth}{5}
\setcounter{secnumdepth}{5}

%-------------------------------------------------------------------------------
% HEADER & FOOTER
%-------------------------------------------------------------------------------
\pagestyle{fancy}
\fancyhf{}
\setlength\headheight{15pt}
\fancyhead[L]{Toledo Margalef, Pablo Adrian}
\fancyhead[R]{Base de Datos II - UNPSJB}
\fancyfoot[R]{\thepage}
%-------------------------------------------------------------------------------
% TITLE PAGE
%-------------------------------------------------------------------------------

\begin{document}

\title{ \normalsize \textsc{Laboratorio III}
    \\ [2.0cm]
    \HRule{0.5pt} \\
    \LARGE \textbf{\uppercase{Postgres, sql:1999, Bases de Datos Relacionales Orientadas a Objetos}}
\HRule{2pt} \\ [0.5cm]}


\author{
    Cátedra: \\
    Lic. Ingravallo, Gabriel\\
    Lic. Parise, Cristian\\
    Integrantes: \\
    Toledo Margalef, Pablo Adrían \\ 
Universidad Nacional de la Patagonia San Juan Bosco}

\maketitle
\newpage

%-------------------------------------------------------------------------------
% Section title formatting
\sectionfont{\scshape}
%-------------------------------------------------------------------------------

%-------------------------------------------------------------------------------
% BODY
%-------------------------------------------------------------------------------

\section*{Consigna Planteada}

\begin{enumerate}
    \item Crear un tipo Aeropuertos que almacene las siguientes propiedades: nombre del aeropuerto, ubicación (ciudad, provincia, pais), medidas de la pista (longitud, ancho, tipo de compuesto) y una coleccion de las aerolineas que trabajan en el mismo.
    \item Crear una tabla aeropuertos basada en el tipo creado en el punto 1. Hacer varios INSERT (y documentarlos) para poblar la tabla aeropuertos con datos.
    \item Crear una subtabla aeropuertosHangares de aeropuertos que refleje aquellos aeropuertos en los que se alquilan hangares que agregue la siguiente informacion: precioEspacio y una coleccion de espacios que registre para cada elemento el nro. de parcela, ocupado (si/no) y una referencia a un avion (objeto de la tabla homónima, deberán considerarse los pasos para tratar a los aviones como objetos OID)

Hacer varios INSERT (y documentarlos) para poblar la tabla aeropuertosHangares con datos.
    \item Resolver las siguientes consultas
        \begin{itemize}
            \item Mostrar todos los aeropuertos que trabajan con la aerolinea X (elegir un valor de X de acuerdo a los datos existentes en la tabla aeropuertos) (uso del ANY)
            \item Mostrar todos los aeropuertos de los cuales se tiene alquilados hangares y que estén ocupados todas las parcelas (uso del ALL) en la consulta deben aparecer los nros. de avion y descripción del modelo de avion que estén ocupando cada parcela.
        \end{itemize}
    \item Desarrole sinteticamente los conceptos de desanidamiento y anidamiento.
    \item Escribir una consulta y documentar el resultado para mostrar en una tabla en 1ra Forma Normal el contenido de la tabla aeropuertosHangares. Verificar que las columnas tengan nombre significativo.
    \item Escribir una consulta para mostrar los nombres de los trabajadores y un arreglo de todos los aviones que repararon mostrando nro de avion y descripcion del modelo, basandose en un JOIN entre las tablas avion, modeloAvion y trabajador.
    \item Escriba en forma resumida una comparacion entre las bases de datos Orientadas a Objetos y las bases de datos Relacionales Orientadas a Objetos
\end{enumerate}

\newpage

%-------------------------------------------------------------------------------
% REFERENCES
%-------------------------------------------------------------------------------

\end{document}

%-------------------------------------------------------------------------------
% SNIPPETS
%-------------------------------------------------------------------------------

%\begin{figure}[!ht]
%   \centering
%   \includegraphics[width=0.8\textwidth]{file_name}
%   \caption{}
%   \centering
%   \label{label:file_name}
%\end{figure}

%\begin{figure}[!ht]
%   \centering
%   \includegraphics[width=0.8\textwidth]{graph}
%   \caption{Blood pressure ranges and associated level of hypertension (American Heart Association, 2013).}
%   \centering
%   \label{label:graph}
%\end{figure}

%\begin{wrapfigure}{r}{0.30\textwidth}
%   \vspace{-40pt}
%   \begin{center}
%       \includegraphics[width=0.29\textwidth]{file_name}
%   \end{center}
%   \vspace{-20pt}
%   \caption{}
%   \label{label:file_name}
%\end{wrapfigure}

%\begin{wrapfigure}{r}{0.45\textwidth}
%   \begin{center}
%       \includegraphics[width=0.29\textwidth]{manometer}
%   \end{center}
%   \caption{Aneroid sphygmomanometer with stethoscope (Medicalexpo, 2012).}
%   \label{label:manometer}
%\end{wrapfigure}

%\begin{table}[!ht]\footnotesize
%   \centering
%   \begin{tabular}{cccccc}
%   \toprule
%   \multicolumn{2}{c} {Pearson's correlation test} & \multicolumn{4}{c} {Independent t-test} \\
%   \midrule    
%   \multicolumn{2}{c} {Gender} & \multicolumn{2}{c} {Activity level} & \multicolumn{2}{c} {Gender} \\
%   \midrule
%   Males & Females & 1st level & 6th level & Males & Females \\
%   \midrule
%   \multicolumn{2}{c} {BMI vs. SP} & \multicolumn{2}{c} {Systolic pressure} & \multicolumn{2}{c} {Systolic Pressure} \\
%   \multicolumn{2}{c} {BMI vs. DP} & \multicolumn{2}{c} {Diastolic pressure} & \multicolumn{2}{c} {Diastolic pressure} \\
%   \multicolumn{2}{c} {BMI vs. MAP} & \multicolumn{2}{c} {MAP} & \multicolumn{2}{c} {MAP} \\
%   \multicolumn{2}{c} {W:H ratio vs. SP} & \multicolumn{2}{c} {BMI} & \multicolumn{2}{c} {BMI} \\
%   \multicolumn{2}{c} {W:H ratio vs. DP} & \multicolumn{2}{c} {W:H ratio} & \multicolumn{2}{c} {W:H ratio} \\
%   \multicolumn{2}{c} {W:H ratio vs. MAP} & \multicolumn{2}{c} {\% Body fat} & \multicolumn{2}{c} {\% Body fat} \\
%   \multicolumn{2}{c} {} & \multicolumn{2}{c} {Height} & \multicolumn{2}{c} {Height} \\
%   \multicolumn{2}{c} {} & \multicolumn{2}{c} {Weight} & \multicolumn{2}{c} {Weight} \\
%   \multicolumn{2}{c} {} & \multicolumn{2}{c} {Heart rate} & \multicolumn{2}{c} {Heart rate} \\
%   \bottomrule
%   \end{tabular}
%   \caption{Parameters that were analysed and related statistical test performed for current study. BMI - body mass index; SP - systolic pressure; DP - diastolic pressure; MAP - mean arterial pressure; W:H ratio - waist to hip ratio.}
%   \label{label:tests}
%\end{table}


